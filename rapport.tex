\documentclass{article}
\usepackage[utf8]{inputenc}
\usepackage{amsmath}
\usepackage{amssymb}

\newcommand{\dx}{\mathrm{dx}}
\newcommand{\dt}{\mathrm{dt}}

\begin{document}

\title{Méthode Numériques: Simulation d'écoulement fluide}
\author{Aurélien Moisson-Franckhauser et Quentin Boyer}
\maketitle

\section{Resolution de l'equation de transport diffusion}

On discretise l'espace en $N_x$ points et le temps en $N_t$ points. On note alors $\mathrm{dx}=\frac{1}{N_x}$ et $\mathrm{dt}=\frac{1}{N_t}$
L'equation devient:


\begin{equation}
	\begin{aligned}
	\phi(x,t+\dt) ={} & \phi(x,t)-c(x)\frac{\dt}{2\dx}(\phi(x+\dx, t)-\phi(x-\dx,t)) \\
					&+c{(x)}^2\frac{\dt^{2}}{2\dx^{2}}(\phi(x+\dx,t)-2\phi(x, t)+\phi{x-\dx}{t}) \\
					&+\kappa\frac{\dt}{\dx^2}(\phi(x+\dx,t+\dt)-2\phi(x,t+\dt)+\phi(x-\dx,t+\dt))
	\end{aligned}
\end{equation}


\paragraph{Question 1}
On note pour $0 \leq k \le N_t$:

\[U^{[k]}=
\begin{bmatrix}
	\phi(0, k\dt) \\
	\vdots \\
	\phi(n\dx, k\dt) \\
	\vdots \\
	\phi((N_x-1)\dx, k\dt)
\end{bmatrix}
\]
On note $U_n^{[k]}$ la composante $n$ de $U^{[k]}$. On cherche alors $ M,N \in \mathcal{M}_{N_x}(\mathbb{R})$ telles que $NU^{[k+1]}=MU^{[k]}$.
En utilisant les notations introduites $(1)$ se reecrit pour $x=n\dx$ et $t=k\dt$:

\[
	\begin{aligned}
		U^{[k+1]}_n ={} & U_n^{[k]} - c(x)\frac{\dt}{2\dx}(U_{n+1}^{[k]}-U_{n-1}^{[k]}) \\
					    & +c{(x)}^2\frac{\dt^{2}}{2\dx^{2}}(U_{n+1}^{[k]}-2U_{n}^{[k]}+U_{n-1}^{[k]}) \\
						& +\kappa\frac{\dt}{\dx^2}(U_{n+1}^{[k+1]}-2U_{n}^{[k+1]}+U_{n-1}^{[k+1]})
	\end{aligned}
\]

Soit

\[
	\begin{aligned}
		&-\kappa\frac{\dt}{\dx^2}U_{n-1}^{[k+1]}+(1-2\kappa\frac{\dt}{\dx^2})U_{n}^{[k+1]}-\kappa\frac{\dt}{\dx^2}U_{n+1}^{[k+1]} \\
		&=c(x)\frac{\dt}{2\dx}(c(x)\frac{\dt}{\dx}+1)U_{n-1}^{[k]} + (1-c{(x)}^2\frac{\dt^{2}}{\dx^{2}})U_n^{[k]} + c(x)\frac{\dt}{2\dx}(c(x)\frac{\dt}{\dx}-1)U_{n+1}^{[k]}
	\end{aligned}
\]

On pose $U_{N_x}^{[k]}=U_{0}^{[k]}$ et $U_{-1}^{[k]}=U_{N_x-1}^{[k]}$.

Cela donne donc 

\[
	N =
	\begin{bmatrix}
		(1-2\kappa\frac{\dt}{\dx^2}) & -\kappa\frac{\dt}{\dx^2} & 0 & \hdots & 0 & -\kappa\frac{\dt}{\dx^2} \\
		-\kappa\frac{\dt}{\dx^2} & (1-2\kappa\frac{\dt}{\dx^2}) & -\kappa\frac{\dt}{\dx^2} & 0\\
		0 & \ddots & \ddots &  \ddots & \ddots \\
		\vdots & \ddots & -\kappa\frac{\dt}{\dx^2} & (1-2\kappa\frac{\dt}{\dx^2}) & -\kappa\frac{\dt}{\dx^2} & 0 \\
		0 \\
		-\kappa\frac{\dt}{\dx^2} & 0 & \hdots & 0 & -\kappa\frac{\dt}{\dx^2} & (1-2\kappa\frac{\dt}{\dx^2})
	\end{bmatrix}
\]
$M$ possede une structure similaire mais avec des coefficients differents:
\[
	M =
	\begin{bmatrix}
		(1-c{(x)}^2\frac{\dt^{2}}{\dx^{2}}) & c(x)\frac{\dt}{2\dx}(c(x)\frac{\dt}{\dx}-1) & 0 & \hdots & 0 & c(x)\frac{\dt}{2\dx}(c(x)\frac{\dt}{\dx}+1)  \\
		c(x)\frac{\dt}{2\dx}(c(x)\frac{\dt}{\dx}+1) & (1-c{(x)}^2\frac{\dt^{2}}{\dx^{2}}) & c(x)\frac{\dt}{2\dx}(c(x)\frac{\dt}{\dx}-1) & 0\\
		0 & \ddots & \ddots &  \ddots & \ddots \\
	\end{bmatrix}
\]
\end{document}
